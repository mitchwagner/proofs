\documentclass{article}
\usepackage{amsmath}
\usepackage{amsthm}
\usepackage{amssymb}

\begin{document}

  \paragraph{Given:} 
    \begin{itemize}
      \item A set of colors, $C$

      \item A directed graph $G=(V,E,w,c)$, where:
        \begin{itemize}
          \item $V$ is the node set of $G$
          \item $E$ is the edge set of $G$
          \item $c: E \rightarrow C$ is a function specifying a color
          for each edge in $G$
        \end{itemize}

      \item A source $s \in V$ and a target $t \in V$ 

    \end{itemize}

  \paragraph{Definitions:} 
    \begin{itemize}
      \item A \textit{color-disjoint} path in $G$ is a path that uses
      at most one edge of each color in $C$.
    \end{itemize}

  \paragraph{Prove:} Finding a color-disjoint $s$--$t$ path in $G$ is
  NP-complete.

  \paragraph{Proof: Reduction of Set Packing}
  \begin{itemize}
    \item \textbf{Given:} an instance $P$ of Set Packing described by:
      \begin{itemize}
        \item A set $S$
        \item A list $L$ consisting of $n$ subsets of $S$
        \item $k \in \mathbb{Z}^+$.
      \end{itemize}

      Set Packing asks whether there exist $k$ pairwise-disjoint
      subsets of $S$ among the $n$ provided.

    \item \label{transform} \textbf{Transformation:} Transform $P$
    as follows: 
      \begin{enumerate}
        \item Define a one-to-one function $m: S \rightarrow C$
        mapping each set element to a unique color. Create a set $C'$ 
        of additional $2k$ colors. 

        \item \label{paths} For each subset $l$ in $L$, define a
        corresponding graph given by a path, as follows: 
          \begin{itemize}
            \item Let each edge of the path correspond to a unique
            element $e$ of $l$.

            \item Assign to each edge the color of its corresponding
            element. 
          \end{itemize}

        \item \label{component} For each path $p$ defined in
        \ref{paths}, create a graph with a source node $a$ connected
        to the start of a copy of $p$, and a target node $b$ connected
        to the end of the copy. Choose a color from the additional
        $2k$ colors in $C'$ and add assign this color to each edge
        incident on $a$. Then, remove the color from $C'$ and add it
        to $C$. Do the same for each edge incident on $b$.

        \item Repeat the process described in \ref{component} $k$
        times. Let $s$ be the source node of the first graph created.
        Let $t$ be the target node of the last graph created. Join all
        $k$ graphs sequentially by defining the target node of the
        first graph as the source node of the second graph, and so on:
        let $G$ be the graph so produced.
        
      \end{enumerate}

    \item \textbf{Implications} 
      \begin{enumerate}
        \item \label{dir1} 
        \textbf{Color-Disjoint Path $\implies$ Set Packing}
        By construction, the existence of any $a$--$b$ path in the
        graph defined by \ref{transform} directly implies the existence
        of $k$ disjoint subsets of $S$.

        \item \label{dir2}
        \textbf{Set Packing $\implies$ Color-Disjoint Path}
        The existence of a color-disjoint path in $G$ is similarly 
        implied by the existence of $k$ pairwise-disjoint subsets in
        $L$. Note that the path will include an additional $2k$ edges
        to account for the traversal from the source node, between
        the intermediate nodes, and to the target node.

        \item \label{done} By \ref{dir1} and \ref{dir2}, we have a reduction
        from Set Packing to the Color-Disjoint Path problem.

        \item By \ref{done}, we have that the task of finding a 
        color-disjoint path in a graph is NP-complete.

      \end{enumerate}

  \end{itemize}
  \qed
    
\end{document}
